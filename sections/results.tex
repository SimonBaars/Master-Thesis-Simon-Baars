\chapter{Results}\label{ch:results}
In this chapter, we present the results of our experiments.

\section{Thresholds} \label{sec:thresholds}
\TODO{todo}

\section{Context Analysis of Clones}\label{chap:clonecontextexpl}
We analyzed the context of clones in a large corpus of open source projects. For these experiments, we used our CloneRefactor tool. These experiments follow the structure of the context: The relation between clone instances is explained, measured and discussed in chapter \ref{chap:relationsinstances}; the location of clone instances is explained, measured and discussed in chapter \ref{chap:clonelocation};  the content of clone instances are explained, measured and discussed in chapter \ref{chap:clonecontents}.

\subsection{Clone detection results}
Currently, we have implemented two clone detection algorithms into CloneRefactor. The first one finds clones by comparing tokens (excluding whitespace, comments and newlines), equal to the definition of type 1 clones in literature \cite{roy2007survey}. The second algorithm implements our type 1R, as explained in chapter \ref{chap:type1clones}. The differences between the clones found for these algorithms is displayed in Table \ref{table:clonedet}.

\begin{table}[H]
  \begin{center}
  \caption{CloneRefactor clone detection results for the two different algorithms.} \label{table:clonedet}
  \medskip
\begin{tabular}{|l|l|l|}
\hline
 & \textbf{Type 1} & \textbf{Type 1R} \\ \hline
\begin{tabular}[c]{@{}l@{}}Amount of lines cloned\end{tabular} & 200,362 & 129,519 \\ \hline
\begin{tabular}[c]{@{}l@{}}Amount of statements/\\declarations cloned\end{tabular} & 182,466 & 118,980 \\ \hline
\begin{tabular}[c]{@{}l@{}}Amount of tokens cloned\end{tabular} & 1,582,845 & 973,596 \\ \hline
\end{tabular}
\end{center}
\end{table}

Looking at Table \ref{table:clonedet}, it becomes apparent that the type 1R algorithm finds significantly less clones than the type 1 algorithm. This indicates that about a third of the clones have textual equality, but are not actually equal when considering the types of expressions. This makes these clones less suitable for automated refactoring.

\subsection{Relations Between Clone Instances} \label{chap:relationsinstances}
In section \ref{sec:setuprelationsinstances} we introduced our experiments regarding relations between clones. Table~\ref{table:relations} contains our results regarding the relations between clone instances. In this table, ``T1'' stands for the type 1 algorithm from literature and ``1R'' stands for our type 1R definition as explained in chapter \ref{chap:type1clones}.

\begin{table}[H]
  \begin{center}
  \caption{Clone relations} \label{table:relations}
  \medskip
\begin{tabular}{|l|l|l|l|l|} \hline
\textbf{Relation}  & \textbf{\# T1} & \textbf{\% T1}  & \textbf{\# 1R} & \textbf{\% 1R} \\ \hline
Unrelated          & 6,134           & 35.48  & 4,762 & 38.14            \\ \hline
\begin{tabular}[c]{@{}l@{}}Same\\Class\end{tabular}          & 4,772            & 27.60  & 3,131 & 25.07             \\ \hline
Sibling            & 2,680            & 15.50   & 1,949 & 15.61            \\ \hline
\begin{tabular}[c]{@{}l@{}}Same\\Method\end{tabular}         & 2,247            & 13.00   & 1,685 & 13.49             \\ \hline
\begin{tabular}[c]{@{}l@{}}External\\Superclass\end{tabular} & 794            & 4.59   & 558 & 4.47             \\ \hline
\begin{tabular}[c]{@{}l@{}}First\\Cousin\end{tabular}        & 269             & 1.56     & 197 & 1.58           \\ \hline
Superclass         & 237             & 1.37     & 118 & 0.94           \\ \hline
\begin{tabular}[c]{@{}l@{}}Common\\Hierarchy\end{tabular}    & 123             & 0.71    & 73 & 0.58            \\ \hline
Ancestor           & 35              & 0.20     & 14 & 0.11          \\ \hline
\end{tabular}
\end{center}
\end{table}

The most notable difference when comparing it to the results of Fontana et al. \cite{fontana2015duplicated} is that in our results most of the clones are unrelated (38.14\% with type 1R), while for them it was only 15.70\%. This might be due to the fact that we consider clone classes rather than clone pairs, and mark the clone class ``Unrelated'' even if just one of the clone instances is outside a hierarchy. It could also be that the corpus which we use, as it has generally smaller projects, uses more classes from outside the project (which are marked ``Unrelated'' if they do not have a common external superclass). About a fourth of all clone classes have all instances in the same class, which is generally easy to refactor. On the third place come the ``Sibling'' clones, which can often be refactored using a pull-up refactoring. There are no noteworthy differences between type 1 and type 1R clones.

\subsection{Clone instance location}\label{chap:clonelocation}
After mapping the relations between individual clones, we looked at the location of individual clone instances. A paper by Lozano et al. \cite{lozano2007evaluating} discusses the harmfulness of cloning. The authors argue that 98\% are produced at method-level. However, this claim is based on a small dataset and based on human copy-paste behavior rather than static code analysis. We validated this claim over our corpus. The results for the clone instance locations are shown in Table \ref{table:locations}. We chose the following categories:
\begin{enumerate}
  \item \textbf{Method/Constructor Level:} A clone instance that does not exceed the boundaries of a single method or constructor (optionally including the declaration of the method or constructor itself).
  \item \textbf{Class Level:} A clone instance in a class, that exceeds the boundaries of a single method or contains something else in the class (like field declarations, other methods, etc.).
  \item \textbf{Interface Level:} A clone that is (a part of) an interface.
  \item \textbf{Enumeration Level:} A clone that is (a part of) an enumeration.
\end{enumerate}

Please note that these results are measured over each clone instance rather than each clone class, hence the higher total amount in comparison to the results of chapter \ref{chap:ourmeasurements}.

\begin{table}[H]
  \begin{center}
  \caption{Clone instance locations} \label{table:locations}
  \medskip
\begin{tabular}{|l|l|l|l|l|}
\hline
\textbf{Location}  & \textbf{\# T1} & \textbf{\% T1}  & \textbf{\# 1R} & \textbf{\% 1R} \\ \hline
Method Level       & 32,861           & 66.02   & 19,075 & 58.23            \\ \hline
Class Level        & 15,069           & 30.27   & 12,207 & 37.27            \\ \hline
\begin{tabular}[c]{@{}l@{}}Constructor\\Level\end{tabular}  & 1,391            & 2.79     & 1,080 & 3.30           \\ \hline
\begin{tabular}[c]{@{}l@{}}Interface\\Level\end{tabular}    & 282              & 0.57     & 247 & 0.75           \\ \hline
Enum Level         & 171               & 0.34    & 147 & 0.45            \\ \hline
\end{tabular}
\end{center}
\end{table}

Our results indicate that around 58\% of the clones are produced at method-level. About 39\% of clones either span several methods/constructors or contain something like a field declaration. Another 3\% of the clones are found in constructors. The amount of clones found in interfaces and enumerations is very low. Regarding the differences between type 1 and type 1R, it seems that there are relatively less method level clones and more class level clones for type 1R. This is probably due to that the main reason for variability between type 1 and type 1R is variable references, which occur more at method level than class level.

\section{Clone instance contents}\label{chap:clonecontents}
Finally, we looked at the contents of individual clone instances: what kind of declarations and statements do they span. We selected the following categories to be relevant for refactoring:
\begin{enumerate}
  \item \textbf{Full Method/Class/Interface/Enumeration:} A clone that spans a full class, method, constructor, interface or enumeration, including its declaration.
  \item \textbf{Partial Method/Constructor:} A clone that spans a method partially, optionally including its declaration.
  \item \textbf{Several Methods:} A clone that spans over two or more methods, either fully or partially, but does not span anything but methods (so not fields or anything in between).
  \item \textbf{Only Fields:} A clone that spans only global variables.
  \item \textbf{Includes Fields/Constructor:} A clone that spans a combination of fields and other things, like methods.
  \item \textbf{Method/Class/Interface/Enumeration Declaration:} A clone that contains the declaration (usually the first line) of a class, method, interface or enumeration.
  \item \textbf{Other:} Anything that does not match with above-stated categories.
\end{enumerate}

The results for these categories are displayed in Table \ref{table:contents}.

\begin{table}[H]
  \begin{center}
  \caption{Clone instance contents} \label{table:contents}
  \medskip
\begin{tabular}{|l|l|l|l|l|}
  \hline
  \textbf{Contents} & \textbf{\# T1} & \textbf{\% T1}  & \textbf{\# 1R} & \textbf{\% 1R} \\ \hline
  \begin{tabular}[c]{@{}l@{}}Partial\\Method\end{tabular}             & 32,214 & 64.72 & 18,791 & 57.37 \\ \hline
  \begin{tabular}[c]{@{}l@{}}Several\\Methods\end{tabular}            & 10,542 & 21.18 & 8,514 & 25.99 \\ \hline
  \begin{tabular}[c]{@{}l@{}}Includes\\Constructor\end{tabular}       & 1,772  & 3.56  & 1,213 & 3.70 \\ \hline
  Includes Field                                                      & 1,681  & 3.38  & 1,487 & 4.54 \\ \hline
  \begin{tabular}[c]{@{}l@{}}Partial\\Constructor\end{tabular}        & 1,389  & 2.79  & 1,078 & 3.29 \\ \hline
  Only Fields                                                         & 962    & 1.93  & 888 & 2.71 \\ \hline
  Full Method                                                         & 647    & 1.30  & 284 & 0.87 \\ \hline
  \begin{tabular}[c]{@{}l@{}}Includes Class\\Declaration\end{tabular} & 263    & 0.53  & 258 & 0.79 \\ \hline
  \begin{tabular}[c]{@{}l@{}}Other\\Categories\end{tabular}           & 304    & 0.61  & 243 & 0.74 \\ \hline
\end{tabular}
\end{center}
\end{table}

Unsurprisingly, most clones span a part of a method. More than a quarter of the clones (for type 1R) span over several methods, which either requires more advanced refactoring techniques or indicates a non-harmful clone.

\section{Merging duplicate code through refactoring} \label{sec:refactorability}
The most used technique to merge clones is method extraction (creating a new method on basis of the contents of clones). However, method extraction cannot be applied in all cases. Sometimes a clone spans a statement partially (like a for-loop of which only it's declaration and a part of the body is cloned). Merging the clones can be harder in such instances. Also, the cloned code can contain statements like \texttt{return}, \texttt{break}, \texttt{continue}. In these instances, more conditions may apply to be able to conduct a refactoring, if beneficial at all.

We measured the amount of clones that can be refactored through method extraction (without additional transformations being required). Our results are displayed in Table \ref{table:refactorability}. In this table we use the following categories:
\begin{itemize}
    \item \textbf{Can be extracted:} This clone is a fragment of code that can directly be extracted to a method. Then, based on the relation between the clone instances, further refactoring techniques can be used to merge the extracted methods (for instance ``pull up method'' for clones in sibling classes).
    \item \textbf{Complex control flow:} This clone contains \texttt{break}, \texttt{continue} or \texttt{return} statements.
    \item \textbf{Spans part of a block:} This clone spans a part of a statement.
    \item \textbf{Is not a partial method:} If the clone does not fall in the ``Partial method'' category of Table \ref{table:contents}, the ``extract method'' refactoring technique cannot be applied.
\end{itemize}

\begin{table}[H]
  \begin{center}
  \caption{Refactorability through method extraction} \label{table:refactorability}
  \medskip
\begin{tabular}{|l|l|l|l|l|}
\hline
\textbf{}        & \textbf{\# T1} & \textbf{\% T1}  & \textbf{\# 1R} & \textbf{\% 1R} \\ \hline
\begin{tabular}[c]{@{}l@{}}Is not a\\partial method\end{tabular}           & 5,917 & 34.22 & 4,806 & 38.49 \\ \hline
\begin{tabular}[c]{@{}l@{}}Complex\\control flow\end{tabular} & 5,511 & 31.87 & 3,158 & 25.29 \\ \hline
\begin{tabular}[c]{@{}l@{}}Spans part of\\a block\end{tabular}             & 3,989 & 23.07 & 3,152 & 25.24 \\ \hline
\begin{tabular}[c]{@{}l@{}}Can be\\extracted\end{tabular}                  & 1,874 & 10.84 & 1,371 & 10.98 \\ \hline
\end{tabular}
\end{center}
\end{table}

From Table \ref{table:refactorability}, we can see that approximately ten percent of the clones can directly be refactored through method extraction (and possibly other refactoring techniques based on the relation of the clone instances). For the other clones, other techniques or transformations will be required. Looking into these techniques and transformations will be one of our next steps.
