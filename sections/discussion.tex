\chapter{Discussion}
\label{ch:discussion}
%Interpretations: what do the results mean?
%Implications: why do the results matter?
%Limitations: what can’t the results tell us?
%Recommendations: what practical actions or scientific studies should follow?

In this chapter, we discuss the results of our research and experiments.
% TODO: Sander - ik mis ergens in chapter nog interpretatie van je resultaten van H6, of heb ik iets gemist?

\section{Clone Type Definitions}
In this study, we proposed a set of clone type definitions for which a refactoring opportunity is known.
% TODO: Sander - is known is beetje gek
% TODO: Sander - zou ook uitleggen waarom we nieuwe definities nodig hadden
We based these clone type definitions on the clone type definitions that are commonly used in literature. The design of these clone type definitions entails some decisions that have a large impact on the clones we considered for refactoring in this study. In this section we discuss our clone type definitions as proposed in Section~\ref{sec:rtypes}.

\subsection{Type 2R clones}
With type 2R clones we allow variability in some identifiers and literals such that the code can and should still be refactored. For type 2R clones we chose a set of expressions in which we allow variability and proposed a recommended refactoring strategy. We think however that type 2R could still use a lot of improvement to find more duplication patterns that can be refactored.

One method we think can be used to find more refactoring opportunities is to allow variability in expressions that have/return the same type. If expressions have/return the same type, they can be extracted to a parameter and the corresponding expression can be passed as a parameter. An example of this is displayed in Figure~\ref{fig:samereturn}. The only thing to watch out for is method that has side effects. Because methods may be executed in another point during execution, this might affect the functionality of the code.
% TODO: Sander - how would you try to solve this?

\begin{figure}[H]
\begin{parcolumns}{2}
\colchunk[1]{
\begin{javacode}
// Original
public void doStuff(){
  int numbers = 456;
|\highlightYellow|  doA(getTitle());
|\highlightYellow|  doB(123);
  doC();
|\highlightYellow|  doA("456");
|\highlightYellow|  doB(numbers);
}

public String getTitle(){
  return "123";
}
\end{javacode}}
\colchunk[2]{
\begin{javacode}
// Refactored
public void doStuff(){
  int numbers = 456;
|\highlightYellow|  doAandB(getTitle(), 123);
  doC();
|\highlightYellow|  doAandB("456", numbers);
}

public void doAandB(String var1, int var2){
  doA(var1);
  doB(var2);
}

public String getTitle(){
  return "123";
}
\end{javacode}}
\end{parcolumns}
\caption{Refactoring different expressions that have the same return type.}
\label{fig:samereturn}
\end{figure}

\section{Clone Context Analysis}
In Section~\ref{chap:contextsetup} we introduced the categories we defined for mapping the context of clones. In this section, we discuss this together with the related experiments.

\subsection{Refactorability} \label{sec:discussrefactorability}
In Section~\ref{sec:refactorability} we introduced to what extent clones can be refactored through method extraction. Because we strived to get results fast, we excluded categories that could not not be directly refactored through method extraction. However, with a few transformations or further considerations it might be possible to make these clones refactorable. In this section we will highlight a few of these categories which we believe to be refactorable through method extraction with a bit more effort.

\subsubsection{Partial block}
We did not consider clones for refactoring that span a part of a block. Although it is indeed not possible to refactor such clones, there are possibilities to make such clones refactorable. For instance, if the programming language supports lambda expressions, we can move the difference of statements in the block in a lambda expression. Figure \ref{fig:partialblockrefactoring} shows an example of such a refactoring opportunity.

\begin{figure}[H]
\begin{parcolumns}{2}
\colchunk[1]{
\begin{javacode}
// Original
public void doStuff(){
|\highlightYellow|  for(int i = 0; i<5; i++) { //Only the declaration of this for loop is cloned, but the loop body is not.
    System.out.println("hello!");
  }
|\highlightYellow|  for(int i = 0; i<5; i++) {
    CoreController.activateCore(i);
  }
}
\end{javacode}}
\colchunk[2]{
\begin{javacode}
// Refactored
public void doStuff(){
|\highlightYellow|  doFiveTimes(() -> System.out.println("hello!"));
|\highlightYellow|  doFiveTimes(() -> CoreController.activateCore(i));
}

public void doFiveTimes(Runnable runnable){
  for(int i = 0; i<5; i++) { //Only the declaration of this for loop is cloned, but the loop body is not.
    runnable.run();
  }
}
\end{javacode}}
\end{parcolumns}
\caption{Refactoring a method that is obstructed by a complex control flow.}
\label{fig:partialblockrefactoring}
\end{figure}

\subsubsection{Complex control flow}
Break, continue and return statements can obstruct the possibility of performing method extraction. However, with some extra transformations, method extraction will still be possible in such cases. Figure \ref{fig:complexcontrolflowrefactoring} shows such a transformation. We can wrap the newly extracted method in a conditional to indicate whether the ``control flow modifying statement'' should be executed. In other cases, other methods might apply to refactor such clones.

\begin{figure}[H]
\begin{parcolumns}{2}
\colchunk[1]{
\begin{javacode}
// Original
public boolean doStuff(){
|\highlightYellow|  if(doA());
|\highlightYellow|    return false;
|\highlightYellow|  doB();
  doC();
|\highlightYellow|  if(doA());
|\highlightYellow|    return false;
|\highlightYellow|  doB();
  return true;
}
\end{javacode}}
\colchunk[2]{
\begin{javacode}
// Refactored
public boolean doStuff(){
|\highlightYellow|  if(!doAandB())
|\highlightYellow|    return false;
  doC();
|\highlightYellow|  return doAandB();
}

public boolean doAandB(){
  if(doA())
    return false;
  doB();
  return true;
}
\end{javacode}}
\end{parcolumns}
\caption{Refactoring a method that is obstructed by a complex control flow.}
\label{fig:complexcontrolflowrefactoring}
\end{figure}

\section{CloneRefactor}
In this section we discuss our decisions for the design of our CloneRefactor tool.

\subsection{Clone Detection}
For CloneRefactor we chose to design a novel method of detecting clones, rather than using an existing clone detection technique or tool. Our rationale is as follows:
\begin{itemize}
  \item We perform comprehensive analysis on the source code which requires us to use an AST-based clone detection method.
  \item We perform dependency graph analysis, which requires us to resolve symbols in the source code.
  \item None of the existing clone detection methods implement all criteria required to build such a system.
\end{itemize}
By building a graph that maps relations between nodes in the AST, we can find clones in an efficient manner, allowing to perform a comprehensive analysis of large systems. This method has worked well for our purposes.

The main limitation we encountered is the memory required to build the clone graph. As we load the entire graph into memory before starting the clone detection procedure, this can cause issues on systems with low available memory. For a system consisting of 1.000.000 nodes, the clone graph requires about 4GB of RAM. For our corpus, there were no larger systems, so this was not a big issue. However, for industry projects our tool might require optimization.

\subsection{Context Analysis}
In this study we identified categories for three properties of clones: relation, location and contents. %Mapping the location and contents was mainly to find out what the best method of refactoring is to refactor most clones.
We chose a set of relations that indicate different refactoring opportunities. However, as our CloneRefactor tool only analyses Java source code, we we biased towards categories that are often found in Java source code. For other languages, other categories might be valuable to analyze to find suitable categories for that programming language.

\subsection{Refactoring}


\subsubsection{Metrics}
For this study we chose to focus on a set of four metrics to measure maintainability: method size, duplication, method parameters and cyclomatic complexity. These metrics give an indication of the impact of the refactoring, but do not give a complete overview. There are many more metrics that could be considered to measure the maintainability impact on the system. An example of such a metric is ``coupling'', which focuses on the amount of incoming calls into a method or class and what modules these calls come from. This metric is also influenced by the transformations we applied and might deliver valuable insights in the quality of the refactoring.

In general, considering other metrics can result in a more reliable measure of the increase or decrease of maintainability after applying a specific refactoring.

\section{Experimental setup}% TODO: Sander - deze sectie is gek, soort combinatie van opzet en threats to validity, het kan wel maar loopt niet super lekker met het geheel
In this section we discuss the setup of our experiments. We discuss alternative setups that could lead to other perspectives on the problem addressed.

\subsection{Java Corpus}
In our experiments we chose to use a GitHub based software corpus. This corpus contains a diversity of projects of many different sizes. We noticed that there is a lot of difference in the quality of these software systems: some systems had a lot of duplication and some did not have any duplication. Furthermore, it took an extensive filtering process to remove all files not suitable for application in this research (like generated files).

We think that there is value in running our experiments with different corpuses. We would, for instance, be interested in the results for industrial projects instead of open source software systems. We are curious to see whether the distributions are comparable, or whether they show large differences.

Furthermore, we think that there is value in running our experiments with a set of larger sized and more professionally developed open source systems. We noticed that in our corpus there were a lot of projects that have only few commits and contributors. We think it would be valuable to, for instance, perform our experiments with a set of larger open source systems, like the systems in the Apache ecosystem.

\subsection{Survey}
To limit the scope of this research, we chose not to include human subjects in this study. However, we think that the results of this study could be strengthened by performing a survey on software engineering practisioners. We have determined a set of refactoring opportunities only based on literature input and quantitive analysis. It would be valuable to have a control group rate the refactorings performed by CloneRefactor, as an extra form of input regarding the quality of the code transformation that CloneRefactor performs.
