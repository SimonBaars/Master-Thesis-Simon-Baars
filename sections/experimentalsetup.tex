\chapter{Experimental Setup}
In this chapter we explain the setup of our experiments. All our experiments are quantitative experiments, measured over a corpus using CloneRefactor. In this chapter, we first explain the corpus we use. We then explain the way in which we calculate the impact of refactorings on the software maintainability.

\section{The corpus}\label{chap:corpus}
For our experiments we use a large corpus of open source projects \cite{githubCorpus2013}. This corpus contains a set of Java projects from GitHub, selected by number of forks. The projects in this corpus were then de-duplicated manually. This results in a variety of Java projects that reflect the quality of average open source Java systems and are useful to perform measurements on.

For our purposes, we have further filtered this corpus to get a controlled set of projects for our experiments. In this section, we explain the steps we took to prepare the corpus and the rationale behind those steps.

\subsection{Project dependencies}
As explained

As indicated in chapter \ref{chap:challenge} CloneRefactor requires all libraries of software projects we test. As these are not included in the used corpus \cite{githubCorpus2013}, we decided to filter the corpus to only include Maven projects. Maven is a build automation tool used primarily for Java, and works on the basis of an \texttt{pom.xml} file to describe the projects' dependencies. As no \texttt{pom.xml} files are included in the corpus, we cloned the latest version of each project in the corpus. We then removed each project that has no \texttt{pom.xml} file. As a final step, we collected all dependencies for each project by using the \texttt{mvn dependency:copy-dependencies -DoutputDirectory=lib} Maven command, and removed each project for which not all dependencies were available (due to non-Maven dependencies being used or unsatisfiable dependencies being referenced in the \texttt{pom.xml} file).

Some general data regarding this corpus is displayed in Table \ref{table:general}.

\begin{table}[H]
  \begin{center}
  \caption{General results for GitHub Java projects corpus \cite{githubCorpus2013}.} \label{table:general}
  \medskip
\begin{tabular}{|l|l|}
\hline
Amount of projects                                                                                      & 1,361      \\ \hline
\begin{tabular}[c]{@{}l@{}}Amount of lines (excluding\\whitespace, comments and newlines.)\end{tabular} & 1,414,996  \\ \hline
Amount of statements/declarations                                                                       & 1,212,189  \\ \hline
\begin{tabular}[c]{@{}l@{}}Amount of tokens (excluding\\whitespace, comments and newlines.)\end{tabular} & 11,643,194 \\ \hline
\end{tabular}
\end{center}
\end{table}
