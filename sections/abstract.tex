\chapter*{Abstract}
\todo[inline,color=blue!10]{This should be done when most of the rest of the document is finished. Be concise, introduce context, problem, known approaches, your solution, your findings.}

Duplication in source code can have a major negative impact on the maintainability of source code. There are several techniques that can be used in order to merge clones, reduce duplication, improve the design of the code and potentially also reduce the total volume of a software system. In this study, we look into the opportunities to aid in the process of refactoring these duplication problems for object-oriented programming languages. We focus primarily on the Java programming language, as refactoring in general is very language-specific.

We first look into redefinitions for different types of clones that have been used in code duplication research for many years. These redefinitions are aimed towards flagging only clones that are useful for refactoring purposes. Our definition defines additional rules for type 1 clones to make sure two cloned fragments are actually equal. We also redefined type 2 clones to reduce the number of false positives resulting from it.

We have conducted measurements that have indicated that more than half of the duplication in code is related to each other through inheritance, making it easier to refactor these clones in a clean way. Approximately a fifth of the duplication can be refactored through method extraction, the other clones require other techniques to be applied.
