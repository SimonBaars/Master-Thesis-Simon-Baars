\chapter{Prioritizing code clones for refactoring}
\label{ch:research_method}
Where a clone instance is located in the code, and how clones in a clone class are related, has a big impact on how this clone should be refactored. Because of this, we have performed measurements on a big corpus of open source projects.

\section{The corpus}
For our measurements we use a large corpus of open source projects \cite{githubCorpus2013}\footnote{The corpus can be downloaded from the following URL: \url{http://groups.inf.ed.ac.uk/cup/javaGithub/java_projects.tar.gz}}. This corpus has been assembled to contain relatively higher quality projects (by filtering by forks). Also, any duplicate projects were removed from this corpus. This results in a variaty of Java projects that reflect the quality of average open source Java systems and are useful to perform measurements on.

We then filtered the corpus further to make sure we are not including any test classes or generated classes. Many Java/Maven projects use a structure where they separate the application and it's tests in the different folders ("/src/main/java" and "/src/test/java" respectively). Because of this, we chose to only use projects from the corpus which use this structure (and had at least a "/src/main/java" folder). To limit the execution time of the script, we also decided to limit the maximum amount of source files in a single project to 1.000 (projects with more source files were not considered). Of the 14.436 projects in the corpus over 3.853 remained, which is plenty for our purposes. The script to filter the corpuses in included in our GitHub repository \footnote{The script we use to filter the corpus: \url{https://github.com/SimonBaars/CloneRefactor/blob/MeasurementsVersion1/src/main/java/com/simonbaars/clonerefactor/scripts/PrepareProjectsFolder.java}}.
