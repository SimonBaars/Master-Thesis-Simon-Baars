To be able to deliver high-quality software without major delays, it is important to control the maintainability of the software. Low maintainability means that developers spend too much time on maintaining and fixing old code. Research shows that maintaining source code takes at least twice as long when maintainability is measured as below average, compared to when maintainability is above average. Lack of maintainability is to a large extent caused by simple issues that occur over and over again. Consequently, the most efficient and effective way to improve maintainability is to address these simple issues.

One of these issues is duplication. Duplication in source code, also called ``code clones'', is the result of a developer reusing code by copying it to another place in the codebase. The often more desired alternative is to, instead of copying code, introduce abstractions to call the same code from different places. Failing to do so needlessly increases the size of the system (up to 25\% volume overhead). It also makes the system more error-prone, because making a change in a cloned code fragment may have to be applied to the other instances as well. Failing to do so can result in unintended behavior.

To solve such duplication issues several refactoring techniques can be used. Refactoring is the process of restructuring code to improve its quality attributes, without changing its functionality. Manual refactoring efforts to reduce duplication take time, require expertise and are prone to cause bugs. To mitigate these issues, we look into to what extent duplication can automatically be refactored.

We first look into how to define clones such that they can be refactored. We look into existing clone type definitions, which argue about the variance that is allowed between code fragments to still consider them clones. The problem towards our goals is that these definitions cannot be be automatically refactored. We propose new clone type definitions that allow for automated refactoring, and propose a method with which to detect clones by these definitions.

To determine what refactoring method to use to refactor clones, we consider their context. We define the context of clones as their relation, location and contents attributes. The relation argues about how each similar fragment in a clone is related to one another, which is important to determine where to place the merged code fragment. The location argues about in what type of place the clone is located in the code: this is used to determine the refactorability of the clone (not every location can be refactored). The contents argue about what code the clone spans: based on this different refactoring techniques may be required.

We then perform statistical measurements on these context categories, to understand which refactoring methods can be used to refactor most clones. We find that most clones can be refactored by moving common code to a new method and placing that method in a location that is accessible by all places the duplicated code was used, then replacing each of these fragments by a call to the new method. This technique is called ``Extract Method''. We apply this technique to automatically refactor clones in a large corpus of open-source Java projects and use it to automatically refactor 28\% of the present duplication.

We then look at which characteristics of the refactored clones have the biggest impact on the maintainability of the system. We find that the size of the clone, measured in number of tokens (a token is the smallest element of a program that is meaningful to the compiler), has the biggest impact on the maintainability of the system. A second factor with a large impact is the amount of data used in both code fragments that are declared outside the scope of the fragment. The latter is currently not used to determine whether a clone should be refactored, however our results show that it is relevant to be considered.


% After 15 years of consulting about software quality, we at the Software Improvement
% Group (SIG) have learned a thing or two about maintainability.
% First, insufficient maintainability is a real problem in the practice of software development. Low maintainability means that developers spend too much time on main‐
% taining and fixing old code. That leaves less time available for the most rewarding
% part of software development: writing new code. Our experience, as well as the data
% we have collected, shows that maintaining source code takes at least twice as long
% when maintainability is measured as below average, compared to when maintainabil‐
% ity is above average. See Appendix A to learn how we measure maintainability.
% Second, lack of maintainability is to a large extent caused by simple issues that occur
% over and over again. Consequently, the most efficient and effective way to improve
% maintainability is to address these simple issues. Improving maintainability does not
% require magic or rocket science. A combination of relatively simple skills and knowl‐
% edge, plus the discipline and environment to apply them, leads to the largest improve‐
% ment in maintainability.
% At SIG, we have seen systems that are essentially unmaintainable. In these systems,
% bugs are not fixed, and functionality is not changed or extended because it is consid‐
% ered too time-consuming and risky. Unfortunately, this is all too common in today’s
% IT industry, but it does not have to be like that.

% Duplication in source code is often seen as one of the most harmful types of technical debt as it increases the size of the codebase and creates implicit dependencies between fragments of code. To remove such anti-patterns, the codebase should be refactored. Many tools aid in the detection process of such duplication problems but fail to determine whether refactoring a duplicate fragment would improve the maintainability.

% We address this shortcoming by combining automated refactoring and maintainability metrics to measure the impact of refactoring code clones. We propose definitions of clones that can be automatically refactored. We then propose a tool to detect such clones, analyze their context and automatically refactor a subset of them. We use a set of established metrics to determine the impact of the applied refactorings on the maintainability of the system. Based on these results, one could decide which clones improve system design and thus should be refactored. We evaluate our approach over a large corpus of open source Java projects.

% We analyze the overhead of applying a refactoring in terms of four factors: the size of the clone, the relation between the code fragments in a clone, whether clone fragments create, modify or return data and the amount of external data that cloned fragments use. We find that the size of similar fragments in a clone is the biggest influencing factor: the majority of duplicates with a total volume of 68 or more tokens improve maintainability when refactored. The amount of external data that needs to be passed to the merged location of the duplicate is the other important factor: the refactored method should not have more than 1 parameter.