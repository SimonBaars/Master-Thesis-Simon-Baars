\documentclass[conference]{IEEEtran}
\IEEEoverridecommandlockouts
% The preceding line is only needed to identify funding in the first footnote. If that is unneeded, please comment it out.
\usepackage{cite}
\usepackage{amsmath,amssymb,amsfonts}
\usepackage{algorithmic}
\usepackage{graphicx}
\usepackage{textcomp}
\usepackage{xcolor}
\def\BibTeX{{\rm B\kern-.05em{\sc i\kern-.025em b}\kern-.08em
    T\kern-.1667em\lower.7ex\hbox{E}\kern-.125emX}}
\begin{document}

\title{Paper Title*\\
{\footnotesize \textsuperscript{*}Note: Sub-titles are not captured in Xplore and
should not be used}
\thanks{Identify applicable funding agency here. If none, delete this.}
}

\author{\IEEEauthorblockN{1\textsuperscript{st} Simon Baars}
\IEEEauthorblockA{%\textit{Software Engineering} \\
\textit{University of Amsterdam}\\
Amsterdam, the Netherlands \\
simon.mailadres@gmail.com}
\and
\IEEEauthorblockN{2\textsuperscript{nd} Ana Oprescu}
\IEEEauthorblockA{%\textit{dept. name of organization (of Aff.)} \\
\textit{University of Amsterdam}\\
Amsterdam, the Netherlands \\
ana.oprescu@uva.nl}
}

\maketitle

\begin{abstract}
Coming up!
\end{abstract}

\begin{IEEEkeywords}
code clones, refactoring, static code analysis, object-oriented programming
\end{IEEEkeywords}

\section{Introduction}
This document is a model and instructions for \LaTeX.
Please observe the conference page limits.

\section{Background}

\subsection{Code clone terminology}
Clone class collection, clone class, clone instance, cloned node, token.

\subsection{Refactoring techniques}
Extract method, move method.

\section{Defining refactorable clones}
Instead of defining clone types, I just define how we can ensure that clones can be refactored. I do not plan to differentiate between T1R, T2R and T3R here, because I think it makes the paper confusing.

To be honest, I'm not even sure how relevant it is to name the literature clone types. Maybe not focus on their shortcomings, but just start from the core (sourcecode) and define how we can get to refactorable clones.

\subsection{Ensuring Equality}
Explain type 1, why it is not always refactorable, solution

\subsection{Allowing variability in a controlled set of expressions}
Explain type 2, why it is not always refactorable, solution

\subsection{Gapped clones}
Explain type 3, why it is not always refactorable, solution

\section{CloneRefactor}
The tool: Detect Clones -> Map Context -> Refactor.

\subsection{Clone Detection}
Detecting refactorable clones, (clone graph?? do I want to explain my exact methods?)

\subsection{Context Mapping}
\subsubsection{Relation}
\subsubsection{Location}
\subsubsection{Contents}

\subsection{Refactoring}
\subsubsection{Extract Method}
Show my categories to show what clones can be dealt with by method extraction.

\subsubsection{AST Transformation}
Explain what AST transformations I do to apply the refactorings.

\subsubsection{The cyclic nature of refactoring}
Explain how refactoring code clones might open up new refactoring opportunities.

\section{Results}

\subsection{Clone context}
How many clones are there in certain contexts?

\subsection{Clone refactorability}
To what extent can found clones be refactored through method extraction, without requiring additional transformations.

\subsection{Thresholds}
I think the ultimate goal with this thesis is to experiment the refactored source code

\section{Discussion}


\section{Conclusion}

\section*{Acknowledgment}
:-)

\section*{References}


\end{document}
