\documentclass[conference]{IEEEtran}
\IEEEoverridecommandlockouts
% The preceding line is only needed to identify funding in the first footnote. If that is unneeded, please comment it out.
\usepackage{cite}
\usepackage{amsmath,amssymb,amsfonts}
\usepackage{algorithmic}
\usepackage{graphicx}
\usepackage{textcomp}
\usepackage{xcolor}
\def\BibTeX{{\rm B\kern-.05em{\sc i\kern-.025em b}\kern-.08em
    T\kern-.1667em\lower.7ex\hbox{E}\kern-.125emX}}
\begin{document}

\title{Improving Software Maintainability through Automated Refactoring of Code Clones}

\author{\IEEEauthorblockN{1\textsuperscript{st} Simon Baars}
\IEEEauthorblockA{%\textit{Software Engineering} \\
\textit{University of Amsterdam}\\
Amsterdam, the Netherlands \\
simon.mailadres@gmail.com}
\and
\IEEEauthorblockN{2\textsuperscript{nd} Ana Oprescu}
\IEEEauthorblockA{%\textit{dept. name of organization (of Aff.)} \\
\textit{University of Amsterdam}\\
Amsterdam, the Netherlands \\
ana.oprescu@uva.nl}
}

\maketitle

\begin{abstract}
Duplication in source code is often seen as one of the most harmful types of technical debt, because it increases the size of the codebase and creates implicit dependencies between fragments of code. To remove such antipatterns, a developer should refactor the codebase. There are many tools that assist in this process. However, these are limited to measuring the size of clones to determine whether they should be considered. Because of this, the output of such tools is often of limited assistance to the developer in the refactoring process.

We propose method and tool for the detection and automated refactoring of clones. We use this tool on a corpus of open source Java systems. We use a set of metrics to determine the impact of the applied refactorings to the maintainability of the systems. These metrics are system size, cyclomatic complexity, duplication and number of method parameters. On basis of these results, we decide which clones improve system design and thus should be refactored.

We identified a set of four factors that influence the maintainability impact of clones. The first is the size of the clone. The second is the relation between the code fragments in a clone. The third is whether the clone fragments create, modify or return data. The fourth is the amount of data that cloned fragments use. By using these four factors, we can suggest only clones that will improve maintainability when refactored and prioritize them accordingly.
\end{abstract}

\begin{IEEEkeywords}
code clones, refactoring, static code analysis, object-oriented programming
\end{IEEEkeywords}

\section{Introduction}
Duplication in source code is often seen as one of the most harmful types of technical debt, because it increases the size of the codebase and create implicit dependencies between fragments of code \cite{ostberg2014automatically}. Bruntink et al.~\cite{bruntink2005use} show that code clones can contribute up to 25\% of the code size.

Current code clone detection detection techniques base their thresholds and prioritization on a limited set of metrics. Often, clone detection techniques are limited to measuring the size of clones to determine whether they should be considered. Because of this, the output of clone detection tools is often of limited assistance to the developer in the refactoring process.



\section{Background}
This study researches an intersection of code clone and refactoring research.

\subsection{Code clone terminology}
Clone class collection, clone class, clone instance, cloned node, token.

\subsection{Code clone definitions}
Quick overview of clone type definitions

\subsection{Refactoring techniques}
In this section, we describe
Extract method, move method.

\section{Defining refactorable clones}
% Instead of defining clone types, I just define how we can ensure that clones can be refactored. I do not plan to differentiate between T1R, T2R and T3R here, because I think it makes the paper confusing.
% To be honest, I'm not even sure how relevant it is to name the literature clone types. Maybe not focus on their shortcomings, but just start from the core (sourcecode) and define how we can get to refactorable clones.
In literature, several clone type definitions have been used to argue about duplication in source code \cite{roy2007survey}. In this section, we discuss how we can define clones such that they can be refactored without side effects on the source code.

\subsection{Ensuring Equality}
% Explain type 1, why it is not always refactorable, solution
Most modern clone detection tools detect clones by comparing the code textually together with the ommission of certain tokens \cite{roy2009comparison, svajlenko2014evaluating}. Clones detected by such means may not always be suitable for refactoring, because textua comparison fails to take into account the context of certain symbols in the code. Information that gets lost in textual comparison is the referenced declaration for type, variable and method references. Equally named type, variable and method references may refer to different declarations with a different implementation. Such clones can be harder to refactor, if benificial at all.

To detect clones that can be refactored, we propose to:
\begin{itemize}
  \item Compare variable references not only by their name, but also by their type.
  \item Compare referenced types by their fully qualified identifier (FQI). The FQI of a type reference describes the full path to where it is declared.
  \item Compare method references by their fully qualified signature (FQS). The FQS of a method reference describes the full path to where it is declared, plus the FQI of each of its parameters.
\end{itemize}

% Do we want an example here?

\subsection{Allowing variability in a controlled set of expressions}
%Explain type 2, why it is not always refactorable, solution
Often, fragments in source code do not match exactly. Often when developers duplicate fragments of code, they modify the duplicated block to fit its new location and purpose. To detect duplicate fragments with minor variance, we looked into in what expressions we can allow variability while still being refactorable.

\subsection{Gapped clones}
Explain type 3, why it is not always refactorable, solution

\section{CloneRefactor}
% NEED TO EXPLAIN WHY DO CURRENT DETECTION TOOLS NOT SUFFICE
The tool: Detect Clones, Map Context, Refactor.

\subsection{Clone Detection}
Detecting refactorable clones, (clone graph?? do I want to explain my exact methods?)

\subsection{Context Mapping}
\subsubsection{Relation}
The rationale for our categories regarding clone relations.

\subsubsection{Location}
The rationale for our categories regarding clone locations.

\subsubsection{Contents}
The rationale for our categories regarding clone contents.

\subsection{Refactoring}
\subsubsection{Extract Method}
Show my categories to show what clones can be dealt with by method extraction.

\subsubsection{AST Transformation}
Explain what AST transformations I do to apply the refactorings.

\subsubsection{The cyclic nature of refactoring}
Explain how refactoring code clones might open up new refactoring opportunities. We refactor a project until there are no more open refactoring opportunities.

\section{Results}

\subsection{Clone context}
How many clones are there in certain contexts? Experiments for relation, location and context.

\subsection{Clone refactorability}
To what extent can found clones be refactored through method extraction, without requiring additional transformations.

\subsection{Thresholds}
I think the ultimate goal with this thesis is to do experiments with different clone thresholds. Which thresholds give clones that we should refactor? For this, we will measure the maintainability of the refactored source code over different thresholds. These thresholds range from minimum clone size, variability and gap size.

\section{Discussion}
\subsection{Clone Definitions}

\subsection{CloneRefactor}

\subsection{Experimental setup}

\section{Conclusion}

\section*{Acknowledgment}
We would like to thank the Software Improvement Group for their continuous support in this project. In particular, we would like to thank Xander Schrijen for his invaluable input in this research. Furthermore, we would like to thank Sander Meester for his proofreading efforts and feedback.

\section*{References}


\end{document}
