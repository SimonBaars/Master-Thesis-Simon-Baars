\documentclass[twocolumn,showpacs,%
  nofootinbib,aps,superscriptaddress,%
  eqsecnum,prd,notitlepage,showkeys,10pt]{article}

\usepackage{amssymb}
\usepackage{amsmath}
\usepackage{graphicx}
\usepackage{dcolumn}
\usepackage{hyperref}
\usepackage{float}

\begin{document}

\title{Statement-level AST-based Clone Detection in Java using Resolved Symbols}
\author{Simon Baars, Ana Oprescu}
%\affiliation{University of Amsterdam}

\maketitle


\begin{abstract}
Duplication in source code is often seen as one of the most harmful types of technical debt as it increases the size of the codebase and creates implicit dependencies between fragments of code. Detecting such problems can provide valuable insight into the quality of systems and help to improve the source code. To correctly identify cloned code, contextual information should be considered, such as the type of variables and called methods.

Comparing code fragments including their contextual information introduces an optimization problem, as contextual information may be hard to retrieve. It can be ambiguous where contextual information resides and tracking it down may require to follow cross-file references. For large codebases, this can require a lot of time due to the sheer number of referenced symbols.

We propose a method to efficiently detect clones taking into account contextual information. To do this, we propose a tool that uses an AST-parsing library named JavaParser to detect clones a retrieve contextual information. Our method first parses the AST retrieved from JavaParser into a graph structure, which is then used to find clones. This graph maps the following relations for each statement in the codebase: the next statement, the previous statement, and the previous cloned statement.
\end{abstract}

\maketitle

\chapter{Introduction}
%\todo[inline,color=blue!10]{Context: what is the bigger scope of the problem you are trying to solve? Try to connect to societal/economical challenges.
%Problem Analysis: Here you present your analysis of the problem situation that your research will address.
%How does this problem manifest itself at your host organisation?
%Also summarises existing scientific insight into the problem.}
\label{ch:introduction}
% TODO: Sander - What is refactoring? Hier zeg je meteen waarvoor het gebruikt wordt
Refactoring is the process of restructuring code to improve quality related attributes of a codebase (maintainability, performance, etc.) without changing the functionality. There are many methods that have been introduced to help with the process of refactoring \cite{fowler2018refactoring, wake2004refactoring}. However, most of these methods still require manual assessment of where and when to apply them. Because of this, refactoring takes up a signification portion of the development process \cite{lientz1978characteristics, mens2004survey}, or does not happen at all \cite{mens2003refactoring}. %For a large part, refactoring requires domain knowledge to do it right. However, there are also refactoring opportunities that are rather trivial and repetitive to execute.
% De zien hierna staat ook al in laatste alineas, past minder goed binnen intro imo
%In this thesis, we take a look at the challenges and opportunities in automatically refactoring duplicated code, also known as ``code clones''. The main goal is to improve maintainability of the refactored code.

Duplication in source code is often seen as one of the most harmful types of technical debt. If the clone is altered at one location to correct an erroneous behaviour, you cannot be sure that this correction is applied to all the cloned code as well \cite{ostberg2014automatically}. Additionally, the code base size increases unnecessarily and so increases the amount of code to be handled when conducting maintenance work, as code clones can contribute up to 25\% of the code size \cite{bruntink2005use}.

In this study, we use refactoring techniques to automatically reduce duplication in software systems. This allows us to obtain before- and after-refactoring snapshots of software systems. We use software maintainability metrics to measure the impact of refactoring clones. We also look into what variability we can allow between code fragments to still consider them clones, while still improving maintainability when refactoring these clones. Futhermore, we look into the thresholds that should be used while detecting clones to find clones that should be refactored.

We perform several quantitative experiments on a large corpus of open source software to collect statistical data. With these experiments we map the context of clones: where they reside in the codebase and what the relation is between duplicate fragments. We use the results to find appropriate refactoring opportunities for specific clones. We then automate the process of applying such refactorings, to be able to measure the impact on maintainability when refactoring clones found by certain definitions and thresholds.

\section{Problem statement}
In this section we describe the problem we address in this study and the research questions that we answer in order to contribute to solving the problem.

\subsection{The problem}
The maintainability of a codebase has a large impact on the time and effort spent on building the desired software system \cite{bakota2012cost, munson1978software}. The maintainability of software is one of the factors to be kept under control in order to avoid major delays and unexpected costs as a result of a software project \cite{fowler2018refactoring}. One factor that has a major impact on the maintainability of a software system is the amount of duplicate code present in a codebase \cite{heitlager2007practical, fowler1999refactoring}.

The process of improving maintainability through the refactoring of duplicate code is time consuming and error-prone. This process mainly consists of these two aspects:
\begin{itemize}
	\item Find refactoring candidates, either tool-assisted or fully manually. % TODO: Sander - is tool assisted niet ook beetje manual? lijkt meer tool-assisted vs fully manual of iets
	\item Refactor identified candidates, either tool-assisted or fully manually.
\end{itemize}
There are tools that assist in the process of finding duplicate code, but none of these tools identify all duplication problems. Often, these clone detection tools result in false-positives and false negatives \cite{roy2007survey}. We define false negatives as clones that can and should be refactored but are not found due to the configuration by which they are detected. We define false-positives as clones that do not improve the system maintainability when refactored. Many false negatives are due to the thresholds that are used. Many clone detection tools \cite{sajnani2016sourcerercc, svajlenko2016bigcloneeval} only consider clones that have a minimum of 6 lines. Using such an approach is bound to result in many false negatives, but using a lower threshold is shown to increase the number of false-positives. %TODO cite threshold paper

A study by Batova et al. \cite{bavota2012does} shows that the process of refactoring often leaves side effects in the code. This study reports that refactorings involving hierarchies (e.g., pull up method) % TODO: Sander -unsure of pull up method bekende term is, miss "creating a new method in a superclass"
, tend to induce faults very frequently. \textit{This suggests more accurate code inspection or testing activities % TODO: Sander - .. are needed?
when such specific refactorings are performed.} Such specific refactorings often have to be used when dealing with duplication in software \cite{fowler2018refactoring, fontana2015duplicated}. Because of that, refactoring code clones has been empirically proven to cause bugs or other side effects in code.

\subsection{Research questions}
There is a lot of research on the topic of code clone detection. This research often results in tools that can detect clones by several clone type definitions. However, there is no research yet that looks into how these definitions align with refactoring opportunities. We will align clone type definitions as used in literature \cite{roy2007survey} with their corresponding refactoring methods \cite{fowler2018refactoring}. For this, we answer the following research question:
\begin{displayquote}
\textbf{Research Question 1:}\\How can we define clone types such that they \textbf{can} be automatically refactored?
\end{displayquote}
As a result, we expect to formulate clone type definitions that can be refactored. On the basis of this results we can perform analyses on the context of clones by these definitions. The context of a clone (location, relation between clones, etc.) has a big impact on how a clone should be refactored. We will create categories by which we map the context of clones and perform a statistical analysis on this. This results in the following research question:
\begin{displayquote}
\textbf{Research Question 2:}\\How can we prioritize refactoring opportunities based on the \textbf{context} of clones?
\end{displayquote}
We expect this research question to result in a priorization of refactoring opportunities: \textit{with what refactoring method can what percentage of clones be refactored?} As a result of these first two research questions we expect to have clone type definitions that can be refactored together with a prioritization of the refactoring methods that can be used. On the basis of this we expect to be able to build a script that can automatically refactor the highest prioritized clones. With this script, we expect to be able to answer our final research question:
\begin{displayquote}
\textbf{Research Question 3:}\\What are the discriminating factors to decide when a clone \textbf{should} be refactored?
\end{displayquote}
Not in all cases will refactoring duplicated code result in a more maintainable codebase. Because of that, we compare the refactored code to the original code and measure the difference in maintainability. To do this, we will use a practical model consisting of metrics to measure maintainability \cite{heitlager2007practical}. Based on this, we can look into what \textit{thresholds} result in better maintainable code when refactored. These thresholds consist of the size of clones and the variability we can allow between cloned fragments to still consider them clones.

\subsection{Research method}
We perform an \textbf{exploratory} study to look into the opportunities to automatically refactor code clones. To do this, we combine knowledge from literature with our own experience to develop definitions for refactorable clones. We also develop a tool to detect, analyze and refactor such clones. Using this tool, we perform \textbf{quantitative} experiments in which we statistically collect information about duplication in open source software. In these experiments we control several variables to see their impact on the results. During this process we explore concepts and develop understanding, because of which decisions in the study design are based upon the results of the experiments.

\section{Contributions}
Many studies report that code clones negatively affect maintainability \cite{heitlager2007practical, monden2002software, juergens2009code, chatterji2013effects}. However, no studies yet show in what cases code clones can reduce maintainability in source code. Refactoring often includes tradeoffs between design alternatives. With some code clones, the refactored alternative is less maintainable than keeping the duplication \cite{kapser2006cloning, aversano2007clones, hotta2010duplicate, kim2005empirical, krinke2007study, saha2010evaluating}. In this study, we analyze the maintainability of refactored code clones, in order to improve the suggestion of code clones that should be refactored. This assists with both the identification and refactoring of code clones.

\subsection{Identification}
There are many tools to detect clones. The goal of most of these tools is to assist developers in reducing duplication in their code, i.e. assisting in the refactoring process. The problem is that these tools have no limited % TODO: Sander -of limited of no, niet beide ;)
insight on the impact of refactoring such clones on the design of the software. In this study, we can analyze a before- and after-refactoring snapshot of the code to determine the impact. A higher maintainability after refactoring increases the support for the clone being a true-positive. This way the results of our study can support the clone identification process.

\subsection{Refactoring}
The tool that results from this research can aid in the process of applying refactorings to clones. The tool will only apply a refactoring if the clone is refactorable and the maintainability of the source code increases as a result of applying the refactoring. The tool applies only refactorings that do not, in any way, influence the functional correctness of the program. Because of this, potential bugs as a result of refactoring can be avoided \cite{bavota2012does}.

\section{Scope}
% TODO: Sander - iets over OOP hier?
In this study, we perform research efforts to be able to detect refactorable clones. We will apply refactoring techniques to a subset of these clones and analyze the maintainability of the resulting source code.

There is a lot of study on how what metrics to consider to measure maintainability. In this study, we focus solely on the practical maintainability model by Heitlager et al. \cite{heitlager2007practical}. We consider the maintainability scores described in that paper as an sufficiently accurate indication of maintainability. This will be used to quantitatively determine the impact on maintainability when applying refactoring techniques to code clones.

Applying refactoring often entails creating new method declarations, class declarations, etc. Each of these declarations needs to have a name. Because the quality of the name is not included in the maintainability metrics we use, finding appropriate names for automatically refactored code fragments is out of the scope for this study. For our automated refactoring efforts, we will use generated names for these declarations.

It is very disputable whether unit tests apply to the same maintainability metrics that applies to the functional code. Because of that, for this research, unit tests are not taken into scope. The findings of this research may be applicable to those classes, but we will not argue the validity.

\section{Outline}
In Chapter~\ref{ch:background} we describe the background of this thesis. In Chapter~\ref{chap:clonetypes} we list shortcomings with clone definitions from literature and propose a set of clone type definitions which can be automatically refactored. % TODO: Sander - Nu zeg je dat de definities automatisch gerefactored kunnen worden
In Chapter \ref{ch:clonerefactor} we propose a tool to detect, analyze and automatically refactor such clones that can be refactored. Using this tool we perform a set of experiments, of which the results are shown in Chapter~\ref{ch:results} and discussed in Chapter~\ref{ch:discussion}. Chapter~\ref{ch:related_work} contains the work related to this thesis. Finally, we present our concluding remarks in Chapter~\ref{ch:conclusion} together with future work.


\input{case}

\input{approaches}

\input{comparison}

\input{analysis}

\chapter{Conclusion}
\label{ch:conclusion}
In the research we have conducted so far we have made three novel contributions:
\begin{itemize}
    \item We proposed a method with which we can detect clones that can/should be refactored.
    \item We mapped the context of clones in a large corpus of open source systems.
    \item We mapped the opportunities to perform method extraction on clones this corpus.
\end{itemize}

We have looked into existing definitions for different types of clones \cite{roy2007survey} and proposed solutions for problems that these types have with regards to automated refactoring. We propose that fully qualified identifiers of method call signatures and type references should be considered instead of their plain text representation, to ensure refactorability. Furthermore, we propose that one should define thresholds for variability in variables, literals and method calls, in order to limit the number of parameters that the merged unit shall have.

The research that we have conducted so far analyzes the context of different kinds of clones and prioritizes their refactoring. Firstly, we looked at the inheritance relation of clone instances in a clone class. We have found that more than a third of all clone classes are flagged unrelated, which means that they have at least one instance that has no relation through inheritance with the other instances. For about a fourth of the clone classes all of its instances are in the same class. About a sixth of the clone classes have clone instances that are siblings of each other (share the same superclass).

Secondly, we looked at the location of clone instances. Most clone instances (58 percent) are found at method level. About 37 percent of clone instances were found at class level. We defined ``class level clones'' as clones that exceed the boundaries of a single method or contain something else in the class (like field declarations, other methods, etc.). Thirdly, we looked at the contents of clone instances. Most clones span a part of a method (57 percent). About 26 percent of clones span over several methods.

We also looked into the refactorability of clones that span a part of a method. Over 10 percent of the clones can directly be refactored by extracting them to a new method (and calling the method at all usages using their relation). The main reason that most clones that span a part of a method cannot directly be refactored by method extraction, is that they contain \texttt{return}, \texttt{break} or \texttt{continue} statements.

\section{Threats to validity}\label{chap:threatstovalidity}
We noticed that, when doing measurements on a corpus of this size, the thresholds that we use for the clone detection have a big impact on the results. There does not seem to be one golden set of thresholds, some thresholds work in some situations but fail in others. We have chosen thresholds that, according to our manual assessment, seemed optimal. However, by using these, we definitely miss some harmful clones.


\section{Future work} \label{sec:future_work}
This study presents a foundation for research in a largely unexplored field of studies: analyzing maintainability through automated software metric refactoring. However, we scratched just the tip of the iceberg regarding all research opportunities in this field. In this section we describe possible extensions to this research, aswell as other research opportunities in this field of studies.

\subsection{Automated Refactoring for more metrics}
In this study we presented evidence regarding the value of applying automated refactoring to analyze the before- and after state of source code and refactored source code. Analyzing these states we were able to analyze the improvement in maintainability after applying certain refactorings. This allows us to better assess thresholds by which maintainability issues in source code are identified. We also get better insights in the costs and values of applying certain refactoring efforts.

For this study we chose to focus only on the automated refactoring of duplication in source code. However, software maintainability depends on more factors and can be measured by more metrics. These factors also have opportunities to automate their refactorings. We think that several similarly sized studies can be conducted to automate the refactoring of other maintainability metrics.

A study by Heitlager et al. \cite{heitlager2007practical} presents several metrics by which the maintainability of source code can be assessed. They propose thresholds that indicate issues with these metrics. Many of these metrics have automated refactoring opportunities. In this section we will focus on several of these metrics to outline their opportunities for automated refactoring.

\subsubsection{Long parameter list}
When multiple parameters are used together in a method, there is an implicit dependency between these parameters: the dependency of being required by that method. If a lot of data hangs around really tight together, they should be made into their own object \cite{fowler1999refactoring, visser2016building}. Guidelines describe to limit the number of parameters per method to at most 4 \cite{visser2016building}.

If a method has many parameters, we can group strongly related parameters into an object. This can be done automatically, but two things must be considered:
\begin{itemize}
  \item How do we determine whether parameters are strongly related?
  \item At what other places is this data used in unison and should thus use the new abstraction?
\end{itemize}
To determine whether parameters are strongly related we must look into at what places they are used in the codebase. We must then define some threshold that denotes the percentage of usages of these variables in which they are used together. If this threshold exceeds a certain amount, we can group them into an object. We must then trace all places in which they are used together and replace the variables by the newly created abstractions.

\subsubsection{Method complexity}
Method complexity refers to the complexity of the logic in a method body. There are several methods to compute method complexity. The most used complexity metric is (MCCabe) Cyclomatic Complexity \cite{visser2016building}, which refers to the amount of independent paths that can be taken though the source code. Another complexity metric that has recently become fairly popular is Cognitive Complexity \cite{campbell2017cognitive}, which attempts to measure the human perceived complexity. Both indicate an aspect of source code maintainability.

Dealing with method complexity can largely be done by method extraction. We extract a part of a complex method to a new method. This way we split the complexity of the original method into separately testable methods. Also, the methods become easier to read.

Refactoring complex methods can largely be done automatically. Also, many of the results of this study can be reused. To assess an automated refactoring opportunity for complex methods, we should assess which parts of methods can be split to end up with parts of similar complexity. For this, our research can be used to assess whether a given piece of code can be extracted to a new method (see section \ref{sec:refactorability}).

We recommend to extend our tool, CloneRefactor, to allow for such capabilities. CloneRefactor already contains a component that calculates the Cyclomatic Complexity of a given method. Using our automated refactored model, identified problems can relatively easily be refactored.

\subsubsection{Method size}
Method size has a strong relation, in terms of refactoring, with Cyclomatic Complexity. Although method size is often related to Cyclomatic Complexity, a study by Landman et al. \cite{landman2016empirical} shows that Cyclomatic Complexity is not redundant to method size. However, they do share a similar method of refactoring. The automated refactoring opportunities described in the previous section also apply to method size.

\subsubsection{Combining the metrics}
Combining the automated refactoring models for each of these metrics can result in a model that ultimately provides significant improvement in the ease of writing well-maintainable source code. In this study, we presented a few tradeoffs that are the result of refactoring code clones. For instance, refactoring code clones with a lot of variability can result in long parameter lists. However, if we can combine this with automated refactoring of long parameter lists, this tradeoff can be mitigated. This way, it is possible to work towards an model of automated refactoring that reduces manual refactoring efforts significantly.

When programming, there are often trade-offs between technical debt and velocity. When a deadline comes near, often software quality in sacrificed to gain velocity \cite{costello1984software, austin2001effects, shah2014global}. Apart from that, a low programmer aptitude can result in low quality code \ref{cheney1984effects}. This is because often forming appropriate abstractions requires time, effort and critical thinking. By introducing these abstractions automatically, this negative impact can (partly) be mitigated.
%When programming, many developers must take decisions between velocity
%Many programmers report frequent decisions in which they decide to take technical debt

\subsection{Identifying more types/definitions for clones}
In this study, we proposed a set of clone type definitions for which a refactoring opportunity is known. We based these clone type definitions on the clone type definitions that are commonly used in literature. The design of these clone type definitions entailed some decisions that have a large impact on the clones we considered for refactoring in this study.

Choosing different

\subsubsection{Type 4 clones}
In section \ref{chap:backgroundclonetypes} we introduced the clone types that are used in literature. Of these clone types, we considered type 1, 2 and 3 for refactoring. We chose not to consider type 4 clones because they appear far less in source code and are more difficult to detect and refactor. However, that does not mean that these clones should not be refactored.

We think that there are relevant research opportunities in refactoring type 4 clones. Type 4 clones are functionally equal pieces of code that are implemented through different implementations. Although functionally equal, type 4 clones might still differ in other aspects. For instance, for type 4 clones, one alternative might be easier to maintain. Also, they can differ (significantly) in performance.

There are interesting research opportunities in automatically choosing the better alternative among type 4 clone instances.

\subsection{Refactoring the harder to refactor clones}

\subsection{Naming of refactored methods/classes/etc.}
In this study, we took naming of refactored entities out of the scope. We applied automated refactoring to duplication, and gave all generated methods, classes and interfaces generated names. For our purposes that was not much of an issue, because of the validation methods used. We validated our approach using the SIG Maintainability model, which does not take naming quality into account. Because of that, the results of our experiments are in no way dependent on the names we give the generated methods, classes and interfaces.

When using our work with the purpose of refactoring assistance, these names will have to be manually provided. However, recent studies allow assistance in this process by generating a name that matches the body of a declaration. If this can be done in a reliable way, we can apply refactorings without any manual steps required.

A study by Allamanis et al. \cite{allamanis2015suggesting} proposes a machine learning model that can suggest accurate method and class names. This study shows promising results towards generating method and class names on basis of their body and context. However, the source code of this study in not available, making it harder to apply their findings to generate class and method names for any software project.

In a recent study by Alon et al. \cite{alon2018code2seq} they propose code2seq. Code2seq is a machine learning model that guesses the name of a method given a method body. This model has been trained on a large set of method bodies and names. The model already shows promising results. The source code of code2seq is publicly available, making it possible to embed this model in any application. Although this model is still far from perfect, combining it with our research could already greatly reduce the manual refactoring effort required. The main deficiency of this model lies in that its limited to the directly avail

\subsection{Looking into other languages/paradigms}
In this study we describe duplicate code refactoring opportunities for object-oriented languages. We built a tool to refactor code clones in Java and used it to run our experiments.

Applying our experiments with other programming languages than just Java might result in valuable results. Refactoring opportunities are greatly dependent on coding conventions, which differ per language. Other languages might result in different results, which might result in different insights regarding the resolution of duplication problems found. We prioritized refactoring efforts based on Java, which might differ from the prioritization that can result from running our experiments with other programming languages.

In this study we focused only on the object-oriented paradigm because of the shared concepts. However, the problems that come with duplication in source code also appears in different programming languages. Because of that, more insights could be obtained when looking into automated refactoring opportunities for other paradigms. For instance, it might be valuable to look into the opportunities to reduce duplication in the functional domain. Dealing with code clones in the functional paradigm is pretty much an unexplored field and might hold valuable results.

\subsection{Running our experiments with a greater diversity of software projects}
Currently

\subsection{Automatically refactoring code that is duplicated with a library}


\bibliographystyle{plain}
\bibliography{references}

\end{document}
