\documentclass[conference]{IEEEtran}

\usepackage{amssymb}
\usepackage{amsmath}
\usepackage{graphicx}
\usepackage{dcolumn}

\usepackage[T1]{fontenc}
\usepackage[utf8]{inputenc}
\usepackage[table]{xcolor}

\usepackage{booktabs}
\usepackage{multirow}
\usepackage{parcolumns}

\usepackage{todonotes}

\usepackage{hyperref}
\usepackage{float}
\usepackage{parcolumns}

\usepackage{listings}
\lstset{language=Java,
  showspaces=false,
  showtabs=false,
  breaklines=true,
  showstringspaces=false,
  breakatwhitespace=true,
  commentstyle=\color{pgreen},
  keywordstyle=\color{pblue},
  numbers = left,
  stringstyle=\color{pred},
  basicstyle=\ttfamily,
%  moredelim=[il][\textcolor{pgrey}]{$$}
  moredelim=[is][\textcolor{pgrey}]{\%\%}{\%\%}
}

\def\checkmark{\tikz\fill[scale=0.4](0,.35) -- (.25,0) -- (1,.7) -- (.25,.15) -- cycle;}

\definecolor{pblue}{rgb}{0.13,0.13,1}
\definecolor{pgreen}{rgb}{0,0.5,0}
\definecolor{pred}{rgb}{0.9,0,0}
\definecolor{pgrey}{rgb}{0.46,0.45,0.48}

\definecolor{lightyellow}{HTML}{fffdcc}
\definecolor{doublelightyellow}{HTML}{fffa9c}
\definecolor{lightgreen}{HTML}{dcffd4}
\definecolor{lightblue}{HTML}{e0ffff}
\definecolor{lightred}{HTML}{ffe0e0}
\definecolor{lightpink}{HTML}{ffe0ff}
\definecolor{lightorange}{HTML}{ffede0}
\definecolor{lightpurple}{HTML}{f0e0ff}
\newcommand{\highlightYellow}{\makebox[0pt][l]{\color{lightyellow}\rule[-4pt]{1\linewidth}{14pt}}}
\newcommand{\highlightGreen}{\makebox[0pt][l]{\color{lightgreen}\rule[-4pt]{1\linewidth}{14pt}}}
\newcommand{\highlightBlue}{\makebox[0pt][l]{\color{lightblue}\rule[-4pt]{1\linewidth}{14pt}}}
\newcommand{\highlightRed}{\makebox[0pt][l]{\color{lightred}\rule[-4pt]{1\linewidth}{14pt}}}
\newcommand{\highlightPink}{\makebox[0pt][l]{\color{lightpink}\rule[-4pt]{1\linewidth}{14pt}}}
\newcommand{\highlightOrange}{\makebox[0pt][l]{\color{lightorange}\rule[-4pt]{1\linewidth}{14pt}}}
\newcommand{\highlightPurple}{\makebox[0pt][l]{\color{lightpurple}\rule[-4pt]{1\linewidth}{14pt}}}
\newcommand{\highlightDarkyellow}{\makebox[0pt][l]{\color{doublelightyellow}\rule[-4pt]{1\linewidth}{14pt}}}

\lstnewenvironment{javacode}[1][]{\lstset{language=Java,escapechar=|,tabsize=2, breaklines=true, xleftmargin=.25in, keywordstyle=\color{pblue},basicstyle=\small,#1}}{}

\begin{document}

\title{Statement-level AST-based Clone Detection in Java using Resolved Symbols}
\author{\IEEEauthorblockN{1\textsuperscript{st} Simon Baars}
\IEEEauthorblockA{\textit{University of Amsterdam}\\
Amsterdam, the Netherlands \\
simon.mailadres@gmail.com}
}
%\affiliation{University of Amsterdam}

\maketitle

\thispagestyle{plain}
\pagestyle{plain}

\begin{abstract}
Duplication in source code is often seen as one of the most harmful types of technical debt as it increases the size of the codebase and creates implicit dependencies between fragments of code. Detecting such problems can provide valuable insight into the quality of systems and help to improve the source code. To correctly identify cloned code, contextual information should be considered, such as the type of variables and called methods.

Comparing code fragments including their contextual information introduces an optimization problem, as contextual information may be hard to retrieve. It can be ambiguous where contextual information resides and tracking it down may require to follow cross-file references. For large codebases, this can require a lot of time due to the sheer number of referenced symbols.

We propose a method to efficiently detect clones taking into account contextual information. To do this, we propose a tool that uses an AST-parsing library named JavaParser to detect clones a retrieve contextual information. Our method first parses the AST retrieved from JavaParser into a graph structure, which is then used to find clones. This graph maps the following relations for each statement in the codebase: the next statement, the previous statement, and the previous cloned statement.
\end{abstract}

\begin{IEEEkeywords}
clone detection, context, java, parsing, static code analysis
\end{IEEEkeywords}

\section{Introduction}
%Most clone detection tools can be configured using thresholds. These thresholds indicate the minimum number of lines, tokens and/or statements that must be spanned for duplicate fragments to be considered clones. Often, such thresholds are intuitively chosen~\cite{li2006cp, roy2009mutation} or based on a quartile distribution of empirical data~\cite{alves2010deriving}. Using the maintainability score we can find support for which thresholds should be chosen to increase the chance to find clones that improve maintainability when refactored.
Duplicate code fragments are often considered as symptoms of bad design~\cite{fowler2018refactoring}. They create implicit dependencies, thus increasing maintenance efforts or causing bugs in evolving software. Changing one occurrence of such a duplicated fragment may require other occurrences to be changed as well~\cite{ostberg2014automatically}. Also, duplicated code has been shown to add up to 25\% of total system volume~\cite{bruntink2005use}, which entails more code to be maintained.

A lot of tools have been proposed to detect such duplication issues \cite{roy2009comparison, svajlenko2014evaluating, sheneamer2016survey}. These tools can find matching fragments of code, but do not take into account contextual information of code. An example of such contextual information is the name of used variables: many different methods with the same name can exist in a codebase. This can obstruct refactoring opportunities.

We describe a method to detect clones while taking into account such contextual information and propose a tool to detect clones taking into account this contextual information. Run this tool over a corpus of 2,267 Java projects. We find that taking into account contextual information results in 11\% less clones than found when not taking this information into account. Manually inspecting a sample the difference, we find that all clones with different contextual information in our sample are less relevant for refactoring.

\section{Background}
We use two concepts to argue about code clones~\cite{roy2007survey}:
\\ \textbf{Clone instance}: A single cloned fragment.
\\ \textbf{Clone class}: A set of similar clone instances.

%\subsection{Clone Types}
Duplication in code is found in many different forms. Most often duplicated code is the result of a programmer reusing previously written code \cite{haefliger2008code, baxter1998clone}. Sometimes this code is then adapted to fit the new context. To reason about these modifications, several clone types have been proposed%. These clone types are described in Roy et al
~\cite{roy2007survey}:\\
\textbf{Type I:} Identical code fragments except for variations in whitespace (may be also variations in layout), and comments.\\
\textbf{Type II:} Structurally/syntactically identical fragments except variations in identifiers, literals, types, layout, and comments.\\
\textbf{Type III:} Copied fragments with further modifications. Statements can be changed, added or removed next to variations in identifiers, literals, types, layout, and comments.
A higher type of clone means that it is harder to detect and refactor. Many studies adopt these clone types, analyzing them further and writing detection techniques for them \cite{sajnani2016sourcerercc, kodhai2010detection, van2019novel}.

\section{Motivating example}
\begin{figure*}
\begin{parcolumns}{2}
\colchunk[1]{
\begin{javacode}
package com.sb.game;

import java.util.List;

public class GameScene
{
|\highlightYellow|	public void addToList(List l) {
|\highlightYellow|		l.add(getClass().getName());
|\highlightYellow|	}

	public void addTen(int x) {
|\highlightYellow|		x = x + 10; // add number
|\highlightYellow|		Notifier.notifyChanged(x);
|\highlightYellow|		return x;
	}
}
\end{javacode}}
\colchunk[2]{
\begin{javacode}
package com.sb.fruitgame;

import java.awt.List;

public class LoseScene
{
|\highlightYellow|	public void addToList(List l) {
|\highlightYellow|		l.add(getClass().getName());
|\highlightYellow|	}

	public void concatTen(String x) {
|\highlightYellow|		x = x + 10; // concat string
|\highlightYellow|		Notifier.notifyChanged(x);
|\highlightYellow|		return x;
	}
}
\end{javacode}}
\end{parcolumns}
\caption{Example of a clone that, although textually equal, is contextually different.}
\label{fig:type1}
\end{figure*}

Most clone detection tools~\cite{kamiya2002ccfinder, semura2017ccfindersw, roy2008nicad, svajlenko2016bigcloneeval, svajlenko2014evaluating} detect clones by textually comparing code fragments (except for whitespace and comments). Although textually equal, method, type and variable references can still refer to different declarations. In such cases, refactoring opportunities could be invalidated. This can make clones less suitable for refactoring purposes, as they require additional judgment regarding the refactorability of such a clone.

Figure~\ref{fig:type1} shows two clone classes. Merging these clone classes is very hard (and likely not desirable), as both cloned fragments describe different functional behavior. The first cloned fragment is a method that adds something to a \texttt{List}. However, the \texttt{List} objects to which something is added are different. Looking at the \texttt{import} statement above the class, one fragment uses the \texttt{java.util.List} and the other uses the \texttt{java.awt.List}. Both happen to have an \texttt{add} method, but apart from that their implementation is completely different.

The second cloned fragment shows how equally named variables can have different types and thus perform different functional concepts. The cloned fragment on the left adds a specific amount to an integer. The cloned fragment on the right concatenates a number to a String.

This shows that not all textually equal clones can easily be refactored.

\section{CloneRefactor}
We develop a tool named CloneRefactor\footnote{CloneRefactor is available on GitHub: \url{https://github.com/SimonBaars/CloneRefactor}} that detects clones that can (relatively) easily be refactored. This tool uses JavaParser~\cite{tomassetti2017javaparser} to

\section{JavaParser}
An important design decision for CloneRefactor is the usage of a library named JavaParser~\cite{tomassetti2017javaparser}. JavaParser is a Java library which allows parsing Java source files to an abstract syntax tree (AST). JavaParser allows to modify this AST and write the result back to Java source code. This allows us to apply refactorings to the detected problems in the source code.

Integrated into JavaParser is a library named SymbolSolver. This library allows for the resolution of symbols using JavaParser. For instance, we can use it to trace references (methods, variables, types, etc) to their declarations (these referenced identifiers are also called ``symbols''). This is useful for the detection of our refactoring-oriented clone types, as they make use of the fully qualified identifiers of symbols.

To be able to trace referenced identifiers, SymbolSolver requires access to not only the analyzed Java project but also all its dependencies. This requires us to include all dependencies with the project. Along with this, SymbolSolver solves symbols in the JRE System Library (the standard libraries coming with every installation of Java) using the active Java Virtual Machine (JVM). %This has a big impact on performance efficiency. This is visible in our results, as displayed in Section~\ref{sec:clonetypeexperiments}.

\section{Conclusion}

\bibliographystyle{plain}
\bibliography{references}

\end{document}
